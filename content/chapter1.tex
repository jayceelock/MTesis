\chapter{Introduction}

\section{Problem Statement}
\label{sec:problem-statement}

Remote controlled model aeroplanes have been in existence for decades, and have been a favourite among hobbyists for almost equally long. However, the past decade has arguably seen the greatest increase in computing power and efficiency since the silicon transistor was first invented. Thanks to the growth in the mobile technology industries, computers have not only grown more powerful, but also smaller, lighter, cheaper and more power efficient. A consequence of this increase in computing power and decrease in size and cost, has made it possible to place small computers onto a model aeroplane and have it fully control, or assist in controlling, the model aeroplane. 

Coupled with the rise in mobile computing, control theory has also reached a point where it has a better understanding of case of the unstable, underactuated plant, and managed to stabilise and control those plants in some cases. One particular case is the multicopter configuration, which resembles a model helicopter with multiple rotors, with the quadcopter configuration holding special interest. 

These autonomous quadcopters are fitted with a multitude of sensors to provide the control system with orientation and localisation data on the quadcopter. These sensors normally include an accelerometer, gyroscope and Global Positioning System (GPS), along with others, such as a barometer, optic flow sensor, magnetometer, etc. The control system uses these sensor readings to keep the quadcopter stable and level while following a human controller's instructions. When loitering, the control system keeps the quadcopter stable and level, while holding the yaw angle and position constant. 

A consequence of the smaller and lighter sensors is that they are often less accurate than their larger, more powerful counterparts. The implication this holds for quadcopters in general can readily be observed from most quadcopters in loiter mode, where there is often significant drift around the quadcopter's set point. The accuracy of the individual sensors are often known or can readily be determined, but due to the mathematical filtering and fusion of the different sensor readings, as well as other operations that the control system may perform on the sensor data, it is difficult to determine the total accuracy of the resulting measurements made by the quadcopter's sensor suite. As a result, the pose (a combination of translation and rotation data) accuracy of an outdoor quadcopter is not yet known. This thesis attempts to find and implement a reliable method to determine the pose error of a typical quadcopter.  

\section{Project Motivation}

For many years, quadcopter's have been the playthings of hobbyists and the researchers studying them. However, the vast improvements that have been made to Unmanned Aerial Vehicles (UAV), which include quad-, hexa- and octocopters for example, and their controllers in recent years have drawn the attention of large corporations, such as Amazon and DHL, who are interested in incorporating UAV's and quadcopters into their respective workforces. 

Similarly, the world's governments have also noticed the increasing commercial and industrial potential that UAV's have, while recognising the dangers that they pose to society if left unregulated, and as such, many governments have moved to regulate and place restrictions on how UAV's may be used. The general trend of the regulations is that UAV's may be flown by a pilot anywhere below an altitude of 120m, 50m away from people and buildings and 10km away from any aerodrome, as stipulated by the~\cite{sacaa-drone-regs}. These regulations allow for autonomous UAV flight, provided the UAV remain within radio line of sight of a human operator that can take over control of the UAV at any time. 

UAV's at the moment are not very safe devices and pose a very serious health hazard to their surroundings and people if handled incorrectly. This, coupled with a general lack of hardware- and software-based collision avoidance capabilities makes aviation authorities very cautious about allowing fully autonomous flight (i.e.\ out of radio line of sight). There are many safety improvements that can be made to UAV's, such as an improved control strategy, hardware improvements, such as rotor shrouds, or other fail-safe and collision avoidance systems. 

Having accurate pose data of a UAV available is crucial to implementing a better control strategy and collision avoidance system. Here the pose refers to a six-dimensional vector describing the UAV's translation and orientation. However, as mentioned in Section~\ref{sec:problem-statement}, UAV's are very bad at estimating its own pose. Therefore, the pose estimation accuracy of a UAV needs to be determined first before any improvements to its control strategy can be made. Also, if the resulting pose accuracy is known, it can be integrated into a collision avoidance model to improve the UAV's performance. 

\section{Existing and Proposed Solutions}

The current state-of-the-art to determining the pose of a UAV in flight is to use an indoor motion tracking system, such as a Vicon system, which uses a set of infrared cameras to track markers placed on an object. However, such systems cannot be used in this case, since the UAV's of interest here have to have access to their GPS coordinates and must therefore be flown in the outdoors. An outdoor pose measurement system is therefore required. 

Outdoor pose measurement systems, such as radar- or laser-based systems, can also be used to perform pose measurements of a UAV.\@ However, these systems normally come at a premium cost and may be subject to external noise sources, such as surrounding trees or birdlife. It was therefore decided to investigate and implement another pose measurement method that is cheap, accurate, repeatable and easy to use. 

The proposed outdoor UAV pose measurement system is based on computer vision techniques, where pose data of an object can be extracted from image or video data. The system will consist of a camera to capture the image data and a computer to perform the data extraction. 

\section{Project Objectives}

The objectives of this research project are given as follows. 

\begin{itemize}
  \item Design and implement a computer vision pose measurement system.
  \item Determine the measurement accuracy of the computer vision system.
  \item Use the computer vision system to determine the pose estimation accuracy of a demonstration quadcopter in flight. 
\end{itemize}

\section{Report Structure}

This thesis document begins with a review of existing literature and of previous research results in the fields relevant to this project, such as computer vision techniques, UAV control strategies, and others. Then, the design and implementation, as well as the determination of the accuracy of the computer vision pose estimation system is discussed, followed by a discussion on how the system was used to determine the pose estimation accuracy of a quadcopter in flight. Finally, a conclusionary discussion on the significant findings and results, as well as potential improvements of this project, is presented. 
