\chapter{Introduction}

\section{Problem Statement}
\label{sec:problem-statement}

Remote controlled model aeroplanes have been in existence for decades, and have been a favourite among hobbyists for almost equally long. However, the past decade has arguably seen the greatest increase in computing power and efficiency since the silicon transistor was first invented. Thanks to the growth in the mobile technology industries, computers have not only grown more powerful, but also smaller, lighter, cheaper and more power efficient. A consequence of this increase in computing power and decrease in size and cost, have made it possible to place small computers onto a model aeroplane and have it fully control, or assist in controlling, a model aeroplane. 

Coupled with the rise in mobile computing, control theory has also developed a better understanding of the unstable, underactuated plant, and managed to stabilise and control those plants in some cases. One particular case is the multicopter configuration, which resembles a model helicopter with multiple rotors, with the quadcopter configuration holding special interest. 

These autonomous quadcopters, referred to as drones henceforth, are fitted with a multitude of sensors to provide the control system with orientation and localisation data. These sensors normally include an accelerometer, gyroscope and Global Positioning System (GPS), along with others, such as a barometer, optic flow sensor, magnetometer, etc. The control system uses these sensor readings to keep a drone stable and level while following a human controller's instructions. When loitering, the control system keeps the drone stable and level, while holding the yaw angle and position constant. 

A consequence of the smaller and lighter sensors is that they are often less accurate than their larger counterparts. A demonstration of what this means for drones in flight can be readily observed from most drones in loiter mode where there is often significant drift around a drone's set point. The accuracy of the individual sensors are often known or can readily be determined, but due to the mathematical fusion of the different sensor readings, as well as other operations that the control system may perform on the data, it is difficult to determine the total accuracy of the resulting measurements made by the drone's sensor suite. As a result, the pose (a combination of translation and rotation data) accuracy of a outdoor drone is not yet known. This thesis attempts to find and implement a reliable method to determine the pose error of a typical drone.  

\section{Project Motivation}

For many years, drones have been the playthings of hobbyists and the researchers studying them. However, the vast improvements that have been made to drones in recent years have drawn the attention of large corporations, such as Amazon and DHL, who are interested in incorporating drones into their respective workforces. 

Similarly, the world's governments have also noticed the increasing commercial and industrial potential that drones have, while recognising the dangers that they pose to if left unregulated, and as such, many governments have moved to regulate and place restrictions on how drones may be used. The general trend of the regulations is that drones may be flown by a pilot anywhere below an altitude of 120m, 50m away from people and buildings and 10km away from any aerodrome, as stipulated by the~\cite{sacaa-drone-regs}. These regulations allow for autonomous drone flight, provided the drone remain within radio line of sight of a human operator that can take over control of the drone at any time. 

Drones at the moment are not very safe devices and pose a very serious health hazard to surrounding people if handled incorrectly. This, coupled with a general lack of hardware- and software-based collision avoidance, makes aviation authorities are very cautious about allowing full autonomous flight (i.e.\ out of radio line of sight). There are many safety improvements that can be made to drones, such as an improved control strategy, hardware improvements, such as rotor shrouds, or other fail-safe and collision avoidance systems. 

Accurate pose data of the drone is crucial to implementing a better control strategy and collision avoidance system. However, as mentioned in Section~\ref{sec:problem-statement}, drones are very bad at estimating its own pose. Therefore,the pose of a drone needs to be determined first before any improvements to its control strategy can be made. Also, if the resulting pose error is known, it can be integrated into a collision avoidance model to improve its performance. 

\section{Existing Solutions}

Currently there are very accurate methods available to determine the pose of a drone in flight. However, these methods must almost exclusively be performed indoors due to the measurement systems' requirements. Since the position estimate of a drone is very reliant on its GPS reading, it is very important that the pose measurement take place in the outdoors to afford a drone a GPS lock, rendering all indoor measurement system useless in this regard. 

Outdoor measurement systems, such as radar- or laser-based systems, can also be used to perform pose measurements of a drone. However, these systems normally come at a premium cost and may be subject to external noise sources, such as surrounding trees or birdlife. It was therefore decided to investigate and implement another pose measurement method that is cheap, accurate, repeatable and easy to use. 

\section{Proposed Solution}

\section{Project Objectives}

\section{Report Structure}
