\chapter{Introduction}

\section{Problem Statement}

Beskryf probleem:
Drone hou posisie sleg, moet gemeet word om beter beheer te kan word

Remote controlled model aeroplanes have been in existence for decades, and have been a favourite among hobbyists for almost equally long. However, the past decade has arguably seen the greatest increase in computing power and efficiency since the silicon transistor was first invented. Thanks to the growth in the mobile technology industries, computers have not only grown more powerful, but also smaller, lighter, cheaper and more power efficient. A consequence of this increase in computing power and decrease in size and cost, have made it possible to place a small computer onto a model aeroplane and have it fully control, or assist in controlling, the aeroplane. 

Coupled with the rise in mobile computing, control theory has also developed a better understanding of the unstable, underactuated plant, and managed to stabilise and control some cases. One particular case is the multicopter configuration, which resembles a model helicopter with multiple rotors, with the quadcopter configuration holding special interest. 

These autonomous quadcopters, referred to as drones henceforth, are fitted with a multitude of sensors to provide the control system with orientation and localisation data. These sensors normally include an accelerometer, gyroscope and Global Positioning System (GPS), along with others, such as a barometer, optic flow sensor, magnetometer, etc. The control system uses these sensor readings and keeps the drone stable and level while following a human controller's instructions. When loitering, the control system keeps the drone stable and level, while keeping the yaw angle and position constant. 

However, it can be seen from a drone in loiter mode that they do not hold their position very accurately, instead drifting around its set point. This is a due to the decreased accuracy that the smaller, cheaper sensors that are often placed onto drones, have. The accuracy of these individual sensors are often known, but as a consequence of the mathematical fusion of the different sensor readings, as well as other complications introduced by the control system, it is difficult to determine the total accuracy of the resulting measurements made by the drone's sensor suite.

\section{Project Motivation}

\section{Existing Solutions}

LIDAR, RADAR, VICON (Y U NO WORK??)

\section{Proposed Solution}

\section{Project Goal}

\section{Report Structure}
