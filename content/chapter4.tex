\chapter{Pose Error Estimation}

\section{Introduction}


The CVS was to be used as a measurement device to accomplish the objective of determining how accurately a quadcopter can estimate its pose, where the pose of an object being a combination of translation and rotation data. Before this can be done, however, it is important to establish how accurate the pose measurements of the CVS are. This was done in Chapter MEME.\@ Here it was found that the pose measurement error, i.e.\ the difference between the true pose and the measured pose, is highly dependant on the pose of the object being tracked relative to the camera. Also, despite using the RANSAC algorithm to filter out outlier data, a large amount of noise is still present within the CVS measurement data set. 

These results prove that it is close to impossible to establish the measurement error for any arbitrary input pose vector, since the measurement error is not constant. The high dimensionality and complexity of the measurement error data also makes this task more complex. 

One avenue that was investigated, was the use of a machine learning model to predict the pose measurement error that the CVS would make, for any arbitrary input pose measurement vector. 

This Chapter, 

\section{Prediction Model Design}

\subsection{Introduction}

The field of machine learning has produced various methods and algorithms over the many years it has been active, each of them having their own characteristics, strengths and drawbacks. For this project, artificial neural networks (ANN) were considered. It has been found that ANN's are inherently good at handling complex, multi-dimensional data models and finding any non-linear relationships [\cite{tu1996advantages}].

Furthermore,~\cite{xie2011comparison} has found that the radial basis function network (RBFN) sub-family of ANN's handle noisy data sets very well, but is slightly less efficient than normal feed-forward networks (FFN) for classification. However, this is a small price to pay to create an accurate prediction model for the CVS's pose measurement error. 

This section describes the steps taken in designing and testing the error prediction model, starting with the training process.

\subsection{Model Training}

The model training phase is a very important part of the model design process, since it is during this phase that the inter-node weighting factors are determined. The output of the prediction model is highly dependant on these inter-node weights.  

First, the training data selection is important. 
\subsubsection{Data Selection}

To train a prediction model that will output accurate results, it is important to select training data across a wide spectrum in the region where the working point is most likely to be. In the case of the CVS, pose measurement vectors that contain as many pose combinations as possible, is desirable. Furthermore, it is also advantageous to have the sample vectors be uniformly distributed in a frequency histogram. Lastly, in the case of ANN's, the size of the training data set plays a significant roll in the accuracy of the training procedure: use too many samples and the model becomes over fitted, failing to capture the non-linear relationships, whereas too few samples fail to characterise the system properly. 

The RBFN uses a supervised learning scheme where the desired output, the true CVS pose measurement error in this case, must be known. Therefore, the training data vectors were selected out of the data acquired from the Vicon test discussed in Chapter MEME, which produced pose measurement data vectors for the CVS, as well as pose measurement errors for each of the measurement vectors of the CVS. 

To select a training data set that shows a wide variation in values while being uniformly distributed, it was opted to randomly sample a collection of vectors and test the distribution of the dimension of interest for uniformity using a $\chi^2$ test. If the dimension is uniformly distributed enough, then that collection of pose vectors is added to the training set. The process is repeated until a uniform data set for each dataimension is found. It was opted to select a total of 15 vectors per dimension, giving a total 90 training pose vectors. 

\subsubsection{Training Process}

Memememe

For the training process, a supervised method is used. Here, the desired output is known and the training process attempts to fit an optimal number of radial functions, or nodes, that best describes the output data set for the given input. 

The SciPy library's\footnote{SciPy v0.13.3} `Rbf' function was used. The function is written in such a way where it does not iteratively adjust its nodes centres to better fit the data. Instead, it sets up a linear seat of equations and solves them linearly. The sets of equations is of the form given in Equation mememe

\begin{equation}
  \label{eq:chap4-rbf-train}
  Rx = y
\end{equation}

In Equation~\ref{eq:chap4-rbf-train}, the matrix $R$ is a matrix containing the euclidian norm distance between each node's centre. To 

Data
Training process - find nodes centres that best cover data, supervised
Training params selection
Verify


\subsection{Model Validation}

\section{Results}

\section{Conclusion}


