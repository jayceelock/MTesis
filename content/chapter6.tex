\chapter{Conclusion}
\label{chap6}

\section{Introduction}

This research project set out to accomplish two things: find a reliable and relatively cheap method if measuring the pose of an object in the outdoors and then use that system to determine the pose estimation accuracy of an unmanned aerial vehicle (UAV) quadcopter in flight in the outdoors. The pose refers to a six-dimensional vector describing an object's position and orientation. 

Previous chapters have covered each aspect of the computer vision system's design, testing and implementation process, as well as live testing with a quadcopter. This chapter provides a brief summary of the project, as well as the key findings and results. The significance of these findings to the existing body of knowledge is also discussed. Finally, the shortcomings of the findings, as well as potential improvements and future work, are discussed. 

\section{Thesis Summary}

Chapter~\ref{chap1} started the document off with an introduction to the problem, motivated the reason why a new outdoor measurement system needed to be developed and what the potential benefits are to knowing how accurately a UAV can estimate its pose.  

A review of the current body of knowledge pertaining to the relevant aspects of this research project was given in Chapter~\ref{chap2}. It was found that a significant amount of research has already been done in stabilising a UAV and having it hold its position in the air using different control strategies, but thus far there is no literature discussing how well an outdoor quadcopter can estimate its pose. Furthermore, it was established that there are well-understood and widely researched computer vision techniques and libraries available that can extract the pose data of an object from an image. Finally, this chapter includes a brief review of the different machine learning techniques available that can be trained to make accurate measurement error predictions for any given pose measurement vector. The different techniques and their strengths and limitations are discussed.  

In Chapter~\ref{chap3}, all of the design aspects of the computer vision system (CVS) was discussed, which include the hardware and software design, as well as the measurement accuracy test and verification process. It was found that the complex nature of the CVS's measurement error necessitates a measurement error estimation model to predict what the measurement error would be for any input pose measurement vector from the CVS.\@ 

Chapter~\ref{chap4} set out to discuss the use of radial basis function neural networks (RBFNN) to estimate the measurement error for any pose measurement vector produced by the CVS.\@ It discusses the training process and how the trained networks' accuracies were verified. Two RBFNNs were trained (one to handle the position dimensions and another for the orientation) which produce satisfactory levels of error with their estimates and can be used with new pose measurement data.

Finally, Chapter~\ref{chap5} is dedicated to discussing how the CVS and the RBFNNs were used in a flight test with a real quadcopter. A number of flight tests were performed with the Solar Thermal Energy Research Group's (STERG) Suncopter quadcopter platform. During these tests, both the CVS and the Suncopter recorded the quadcopter's pose information. The quadcopter's pose measurement accuracy was determined for all the dimensions, except for the $yaw$ dimension. In this case it was found that the quadcopter produces more accurate measurements than the CVS does. The CVS's error boundaries can still be used as a worst-case measure of error for the quadcopter's yaw estimates. 

%It was found that the Suncopter's pose estimate fell within the error boundaries of the CVS, meaning that it gives more accurate pose estimates than the CVS.\@ However, the error boundary of the CVS can still be used as a worst-case pose estimate error measure for the quadcopter's pose estimates. 

\section{Findings and Contributions to Body of Knowledge}

The objectives of this research thesis are discussed in Chapter~\ref{chap1}. They are stated here again for convenience and are as follows:

\begin{itemize}
  \item Design and implement a relatively cheap computer vision pose measurement system.
  \item Determine the measurement accuracy of the pose measurement system.
  \item Use the computer vision system to determine the pose estimation accuracy of a demonstration quadcopter in flight. 
\end{itemize}

For the first objective, a computer vision-based system was made which uses a single camera to record video data of a calibration object and the OpenCV library to extract six-dimensional pose data of the object. Such a system was required since indoor measurement tools cannot be used to determine an outdoor quadcopter's pose estimation error. It was decided to investigate whether a computer vision system is a viable outdoor measurement solution since existing outdoor measurement systems are very expensive and it was desirable to investigate whether a cheaper alternative could provide the same functionality. The computer vision system is fairly simple, cheap, easy to use and set up. Preliminary tests have shown that its estimates were within the ballpark, however, its pose measurement accuracy had to be accurately determined before it could be used to perform any pose measurements. 

The measurement system's pose accuracy was determined by comparing it with the ground-truth pose measurements of a state-of-the-art Vicon motion tracking system. The measurement error is fairly complex and high-dimensional and the measurement test has shown that it is highly interdimensionally dependant (as shown by the covariance matrix in Equation~\ref{eq:covariance-matrix} and the plots in Figure~\ref{fig:err-contour}). This means that the CVS's pose measurement accuracy is dependant on the calibration object's pose relative to the CVS's camera. Therefore, another method of predicting the CVS's pose measurement error for any given pose measurement vector was required.

It was decided that two RBFNNs would be used to perform the measurement error estimation: one for the position dimensions and another for the orientation. This was done to accomodate the inherent differences between position and orientation data. A RBFNN network type was selected based on its proven ability to work well with noisy input data and detect non-linear relationships between the input dimensions. The RBFNNs were trained with a uniformly distributed data set and validated with randomly selected data points. Both data sets were selected from the same Vicon test. The networks produce acceptably small mean square errors for the validation set. It was found that the RBFNNs perform better in the position dimensions than in the orientation dimensions, with a possible reason being the large amount of noise within the orientation data. 

Finally, the CVS and RBFNNs were used together to determine the pose estimation accuracy of a quadcopter in flight. This was done by performing flight tests with a real quadcopter in the outdoors. The results of the test flight gave an indication of the pose estimation accuracy of a quadcopter. It can be seen from Figure~\ref{fig:chap5-err-hist} that the position dimensions have a standard deviation from the CVS's measurements of approximately $\SI{150}{\mm}$ for the position dimensions and $\ang{3.27}$ in the $roll$ dimension and $\ang{1.9}$ in the $pitch$ dimension. The quadcopter's yaw estimate was found to be more accurate than the CVS's. However, the CVS's error boundary can be used as an expected error measure in this case. These results can be incorporated into a quadcopter to improve its control strategy and evaluate the performance of current controllers. 

%The results of the flight tests show that the quadcopter's own pose estimate is more accurate than the CVS's, which implies that the CVS's error measure can be used as a worst-case error measure for the quadcopter's pose estimate, since the CVS's error has very accurately been determined before. This result can be incorporated into a quadcopter to improve its control strategy.

The work done for this project have been presented in part at the SolarPACES 2015 conference in Cape Town. A proceedings article has also been accepted and is pending publication~\citep{lock2015}. 

\section{Shortcomings and Future Work}

The main research objectives and goals that have been set for this thesis have been achieved. However, there are some issues that have been encountered during the project. There is also potential to take this research further to improve and implement the results. These shortcomings and a few potential avenues of further research are mentioned and discussed here. 

\subsection{Improved Measurement System Accuracy}

In this project a computer vision-based pose measurement system (CVS) was designed and tested. It was found that it produces fairly accurate measurement results and, in conjunction with a neural network measurement error estimator, can be used to measure the pose of a chessboard pattern calibration board. This is a first design iteration and there are several improvements that can be made to improve the system's accuracy. 

One change to the CVS which can be implemented is to use high-definition (HD) video data. It is expected that using an HD camera would improve the CVS's pose measurements, since there are more pixels available and the CVS would be able to determine the chessboard's corner coordinates with a finer degree of accuracy. In this project, low resolution ($640\times480$ pixels) video data was used. The reason for using the lower resolution during the system design phase is that the pose estimation for HD video data took approximately 3.5 times longer than for the low resolution video data (approximately 6.5 hours). This is due to the higher number of pixels in the HD video (3 times more pixels in this case). Due to this long processing time and since it was anticipated that the same program will be run many times throughout the CVS's design process, it was opted to use the low resolution video data instead. The HD upgrade is ready to be implemented, since the CVS's current camera has an HD resolution option of $720\times1280$ pixels. 

%As part of this research project, the CVS's measurement accuracy was determined. It was found that it gives fairly accurate pose estimates in its position measurements, but has large deviations from the true pose within its orientation measurements. This is demonstrated in Figure~\ref{fig:err-norm}, where large standard deviations in the error data are shown, as well as in Figure~\ref{fig:chap4-rbf-valid}, where the RBFNN outputs a large error band in the orientation data. This is a serious shortcoming of the CVS, since this high level of error effectively renders the orientation data unusable. 

%An inspection of the CVS's raw orientation data reveals that it remains fairly consistent, but there is significant noise present in the data, sometimes deviating as much as $\ang{20}$ from the true orientation. This noise would explain the large standard deviations observed in the data. Similarly, the RBFNN will be equally affected, since the training error data will also contain the noise and the network will attempt to fit the model to the noisy data during training and replicate it when presented with a new input data set. 

%The RANSAC algorithm has already been employed to remove outlier data points. However, if another way can be found to smooth or filter out the noisy components, it would greatly improve the accuracy of the CVS in general, especially in the orientation dimensions. 

\subsection{Gaps in the Pose Measurement Data}

It was observed with the flight test data that the CVS's pose measurements contain gaps where no data was recorded. Inspection of the video material showed that for significant sections of the video, OpenCV's corner detection algorithm could not detect all of the expected corners on the calibration board. 

There are a few reasons why this may have occurred. One is that the board was simply too far away and the detector could not get a good enough estimate of the corners. Another is that the uneven lighting conditions in the field test led to the contrast not being great enough to highlight the difference between the white and black tiles on the board. This was somewhat remedied by using an adaptive threshold filter in the corner detector. However, there are still some gaps within the data set. 

A suggestion to permanently fix this issue would be to ensure that the calibration board be evenly lit throughout the test. One possibility would be to perform the test at night time and use spot lights to control the lighting level on the calibration board. 

\subsection{Improved Sensor Hardware}

One avenue of improvement, unrelated to the CVS, which may increase the pose measurement accuracy of a quadcopter significantly, is to improve the sensor hardware it is equipped with. More specifically, if a more accurate GPS sensor can be used, it should show a significant improvement in a quadcopter's localisation results. An improved state estimator and sensor fusion solution may also produce significantly better results. 

Currently, the Suncopter comes equipped with a standard uBlox GPS module, which has an expected accuracy of within $\SI{3}{\m}$ (approximately the normal level of accuracy for a standard GPS module). However, recent improvements in the field of differential GPS technology, especially real-time kinematic (RTK) GPS's, have led to smaller, more accurate GPS units becoming available for use with a quadcopter. There already has been some development done in implementing the Piksi RTK GPS in the Pixhawk controller. The Piksi GPS has a reported accuracy level of within a few centimetres, which is a significant improvement over the traditional GPS modules. 

Other sensor improvements can be made, such as a more accurate magnetometer, inertial measurement unit (IMU), and so on. However, it is not expected that these sensors will have as significant an effect on the quadcopter's localisation ability as adding a differential GPS unit would. 

\subsection{Implement Results}

The pose estimation accuracy of a quadcopter has been determined in Chapter~\ref{chap5} and is ready to be implemented in a real drone. These errors can be incorporated into a quadcopter's controller as an error term which will improve the controllers performance and safety margins. 

Implementing these results will also help several other industries. For example, STERG are looking into using a quadcopter to calibrate heliostat mirrors on a concentrating solar power plant. In this context it is very important to have accurate position data available, or at least know what the expected measurement error is so that it can be taken into account during calibration. 

%The pose estimation accuracy of a quadcopter could not definitively be determined, since it was found that the quadcopter that was tested gave more accurate pose measurements than the CVS did. However, the CVS's pose measurement accuracy can still be used as a worst-case pose estimation measure for a quadcopter, since the quadcopter's pose estimates fall within the established error band of the CVS.\@ 

%With the pose estimation error of a quadcopter now determined, it can be implemented into a quadcopter's control system. This will help to produce a more accurate and stable quadcopter. Furthermore, the CVS can be used to test new quadcopter systems and their control configurations. 

\section{Conclusion}

This project has designed and tested a computer vision-based pose measurement system (CVS) and used to determine the pose estimation accuracy of an outdoor quadcopter in flight. Despite being in its first design iteration, the CVS performed well and produces accurate measurement results which were used to determine a quadcopter's pose estimation accuracy. This accuracy measure was not available previously and can be implemented into a real quadcopter to evaluate and improve its performance. This would be a great advancement into making quadcopters safer and make them more attractive to governments and industry alike. 

%Despite the high hopes that were placed on being able to use a computer vision-based pose measurement system to accurately determine the pose estimation accuracy of a quadcopter in flight, it has been shown that it is not yet accurate enough. However, even in its shortcoming it has provided an error measure for a quadcopter's pose estimates which was not available previously. This error term can be implemented into a quadcopter, creating a `bubble of uncertainty' around the quadcopter. This would be great advancement into making quadcopters safer to use and make it more attractive to government and industry alike. 

It is hoped that these findings will help in improving quadcopters and help the technology live up to its massive potential. 
