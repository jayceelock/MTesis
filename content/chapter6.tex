\chapter{Conclusion}

\section{Introduction}

This research project set out to accomplish two things: find a reliable, accurate and cheap method to measure the six-dimensional pose of an object in the outdoors, and then to use that system to determine the pose estimation accuracy of an unmanned aerial vehicle (UAV) quadcopter in flight in the outdoors. The pose refers to a six-dimensional vector describing an object's position and orientation. 

Previous chapter in this paper have covered each aspect of the computer vision system design, testing and implementation process, as well as live testing with a quadcopter. This chapter will provide a brief summary of this paper, as well as the key findings and results. The significance of these findings to the existing body of knowledge is also discussed. Finally, the shortcomings of the findings and their potential solutions and future work is discussed. 

\section{Thesis Summary}

chapter 1 

Chapter meme starts th e thesis document off with an introduction to the problem and motivates the reason why a new outdoor measurement system needs to be developed and what the potential benefits are to knowing how accurately a UAV can estimate its pose.  

chapter 2

A review of the current body of knowledge pertaining to the relevant aspects of this research project is given in Chapter meme. It was found that a significant amount of research has already been done in stabilising a UAV and having it hold its position in the air with different control strategies. Furthermore, it was established that the computer vision techniques and libraries available that can extract the pose data for an object from an image, are fairly well understood and widely researched. Finally, this chapter includes a brief review of the different machine learning techniques available that can be trained to make accurate predictions for a given input vector. The different techniques with their strengths and shortcomings are discussed.  

chapter 3

In Chapter meme, all of the design aspects of the computer vision system (CVS) is discussed, which include the hardware and software design as well as the measurement accuracy test and verification process. 

chapter 4



chapter 5

\section{Findings and Contributions to Body of Knowledge}

\section{Shortcomings and Future Work}

\section{Conclusion}
