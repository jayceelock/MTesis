\chapter{Conclusion}

\section{Introduction}

This research project set out to accomplish two things: find a reliable, accurate and cheap method to measure the six-dimensional pose of an object in the outdoors, and then to use that system to determine the pose estimation accuracy of an unmanned aerial vehicle (UAV) quadcopter in flight in the outdoors. The pose refers to a six-dimensional vector describing an object's position and orientation. 

Previous chapter in this paper have covered each aspect of the computer vision system design, testing and implementation process, as well as live testing with a quadcopter. This chapter will provide a brief summary of this paper, as well as the key findings and results. The significance of these findings to the existing body of knowledge is also discussed. Finally, the shortcomings of the findings and their potential solutions and future work is discussed. 

\section{Thesis Summary}

Chapter meme starts th e thesis document off with an introduction to the problem and motivates the reason why a new outdoor measurement system needs to be developed and what the potential benefits are to knowing how accurately a UAV can estimate its pose.  

A review of the current body of knowledge pertaining to the relevant aspects of this research project is given in Chapter meme. It was found that a significant amount of research has already been done in stabilising a UAV and having it hold its position in the air with different control strategies. Furthermore, it was established that the computer vision techniques and libraries available that can extract the pose data for an object from an image, are fairly well understood and widely researched. Finally, this chapter includes a brief review of the different machine learning techniques available that can be trained to make accurate predictions for a given input vector. The different techniques with their strengths and shortcomings are discussed.  

In Chapter meme, all of the design aspects of the computer vision system (CVS) is discussed, which include the hardware and software design as well as the measurement accuracy test and verification process. It was found that the complex nature of the CVS's measurement error necessitates a prediction algorithm to predict what the measurement error would be for any given input pose measurement vector from the CVS. 

Chapter meme is set out to discuss using a radial basis function neural network (RBFNN) to estimate the measurement error for the pose vector produced by the CVS. It discusses the training process used and how the trained network's estimation accuracy was was verified. A RBFNN was trained which produces an nacceptable level of error with its estimates and can be used with new pose measurement data.

Finally, Chapter meme is dedicated to discussing how the CVS and the RBFNN was used with a flight test with a real quadcopter.

\section{Findings and Contributions to Body of Knowledge}

The objectives of this research thesis are discussed in Chapter meme. They are stated here again for convenience and are as follows:

\begin{itemize}
  \item Design and implement an alternative outdoor pose measurement system.
  \item Determine the measurement accuracy of that pose measurement system.
  \item Use the computer vision system to determine the pose estimation accuracy of a quadcopter in flight. 
\end{itemize}

The main objective is to determine the pose estimation accuracy of a quadcopter in flight in an outdoor environment. To be able to do this, a new pose measurement tool needed to be employed that would work in the outdoors. To this end, a camera-based system was made that makes use of existing computer vision techniques to extract pose information from image data. However, before this system could be used to make real pose measurements, its accuracy first needed to be determined.

This was done by using a Vicon motion-tracking system, whose measurements provided a ground-truth baseline with which the CVS's pose measurements could be compared against. Consequently, the CVS's measurement error data was found to be complex and high-dimensional and highly interdimensionally dependant, as shown by the covariance matrix in Equation meme. This made it necassary to use a machine learning model to predict measurement error for any input pose measurement vector, since the measurement error is not constant and varies according to the current pose of the object relative to the CVS's camera. 

To perform the measurement error prediction, a RBFNN was selected based on its ability to work well with noisy input data and detect and characterise non-linear relationships between the input dimensions. This network was trained and validated with data gathered during the Vicon test and produces a normalised mean square error of meme for the training set and meme for the validations set. These small errors indicate that the network was trained well and overfitting did not occur. The trained RBFNN was then ready predict the measurement error for new pose measurement data.

With the CVS designed and ready to make measurements and the RBFNN trained and ready to determine the error in the CVS data, a real flight test could be conducted where the pose estimation accuracy of a quadcopter in flight was determined. For the 

\section{Shortcomings and Future Work}

\section{Conclusion}


